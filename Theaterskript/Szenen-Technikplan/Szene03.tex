% This file is part of TenSing-Moers/Theaterskript2018.
%
% TenSing-Moers/Theaterskript2018 is free content: you can redistribute and/or
% modify it under the terms of the cc-by-nc-sa (Creative Commons
% Attribution-NonCommercial-ShareAlike) as released by the
% Creative Commons organisation, version 4.0.
%
% TenSing-Moers/Theaterskript2018 is distributed in the hope that it will be useful,
% but without any warranty.
%
% You should have received a copy of the cc-by-nc-sa-license along
% with this copy of TenSing-Moers/Theaterkskript2018. If not, see
% <https://creativecommons.org/licenses/by-nc-sa/4.0/legalcode>.
%
% Copyright TenSing Moers and all whose work and <3 went in this project.
\section{Die Aufführung}
\DisplayPersons
\DisplayRequisites
\ort{In der Schule}

\StageDir{Generalprobe für die Schulaufführung. \pikachuh und andere Personen laufen herum und sind in Vorbereitungen vertieft. \miladh kommt auf die Bühne.}
\licht{gelb, orange (neutral gehalten)}
\mikroan{1}
\mikroan{2}
\mikroan{3}

\theox{Da bist du ja endlich, Mila! Wo hast du denn so lange gesteckt? Ich hatte schon befürchtet, du kommst nicht mehr, dabei ist deine Rolle doch SO wichtig für die Aufführung! Komm schnell, wir haben nur noch wenig Zeit!}

\StageDir{\theoh zieht \miladh am Arm mit sich, sie folgt ihm widerwillig.}

\theox{Du musst mir nur kurz helfen, dieses große Teil zu verschieben!}
\miladx{Ehm, okay, aber\dots\ Was ist das denn?}

\StageDir{\alberth springt hinter der Maschine hervor, \miladh weicht erschrocken zurück.}

\albertx{Das, wenn ich bitten darf, ist mein Weltreiseapparat! Damit kann man in fremde Welten reisen! Diese ungebildete Lehrkraft hält das aber für zu absurd, und will es deshalb lieber als Requisit für die Aufführung benutzen.}
\miladx{Achso, interessant. Und wer bist du noch gleich?}
\albertx{\direct{besserwisserisch} Ich dachte schon, du fragst gar nicht mehr. Ich bin Albert Einstein, und gehe in die 10. Klasse!}
\miladx{Ach stimmt, von dir habe ich schonmal gehört. Dieser Mathefreak, der keine Freunde hat, respektlos zu den Lehrern ist und sich überhaupt für den allerklügsten hält\dots}
\albertx{\direct{entsetzt} Wie bitte? Diese Worte können nur von jemandem stammen, der die Mathematik nicht wertschätzt! \direct{abfällig} Pf, das sieht man dir doch schon an!}
\miladx{\direct{genervt} Du hältst dich doch auch für was besseres. Nur, weil ich keine bekloppten Erfindungen baue, heißt das nicht, dass ich dumm bin. Ich habe andere Stärken! Und dein blöder Weltreiseapparat funktioniert doch eh nicht.}
\albertx{\direct{schockiert} Wie bitte?! Von einem Mädchen, das nichtmal im Entferntesten meinem Intellekt entspricht, lasse ich mir doch gar nichts sagen! Ha, du wirst schon noch sehen, was du davon hast!}

\mikroaus{1}
\mikroaus{2}
\mikroaus{3}
\StageDir{\alberth schubst \miladh in die Maschine. \alberth stolpert dabei, ein \lichtblitzd erscheint und ein \knalld ertönt, und beide verschwinden.}
\sound{Knall}
\licht{Lichtblitz}