% This file is part of TenSing-Moers/Theaterskript2018.
%
% TenSing-Moers/Theaterskript2018 is free content: you can redistribute and/or
% modify it under the terms of the cc-by-nc-sa (Creative Commons
% Attribution-NonCommercial-ShareAlike) as released by the
% Creative Commons organisation, version 4.0.
%
% TenSing-Moers/Theaterskript2018 is distributed in the hope that it will be useful,
% but without any warranty.
%
% You should have received a copy of the cc-by-nc-sa-license along
% with this copy of TenSing-Moers/Theaterkskript2018. If not, see
% <https://creativecommons.org/licenses/by-nc-sa/4.0/legalcode>.
%
% Copyright TenSing Moers and all whose work and <3 went in this project.
\section{Liebes Tagebuch\dots}
\DisplayPersons
\DisplayRequisites
\ort{In \milamhh 's Zimmer}

% Mit F1 ausführen (oder oben auf "Schnelles Übersetzen" klicken)
% Verfügbare Kommandos:
%\mikroan{1}
%\mikroaus{1}
%\sound{Einspieler-Name}
%\soundwide{Langer Einspieler-Name}
%\licht{Lichtanweisung}

\StageDir{\milamh sitzt am Bühnenrand. Im Hintergrund stellen \lenah, \leonh, \mutterh und Milas Vater pantomimisch dar, was erzählt wird. Es wird ein \plakatad hochgehalten, auf dem \glqq 1894\grqq geschrieben steht. Spot auf \milamh.}

\licht{Spot, Bühnenkante}
\licht{gelb, orange (neutral gehalten)}
\mikroan{1}

\milamx{Liebes Tagebuch, heute war schon wieder so ein doofer Tag! Schon beim Frühstück wollte Mama von mir, dass ich mein Zimmer aufräume, und \leonh hat dabei die ganze Zeit genervt und wollte mit mir spielen. Geschenke habe ich immer noch keine besorgt. Dabei ist schon in einer Woche Weihnachten! Und zu allem Überfluss ist es der Schule momentan auch alles andere als spannend. Mathematik, Biologie, Physik... Wer braucht das schon?! Papa ist deswegen, glaube ich, auch ein bisschen enttäuscht, und meint ständig, ich soll mir mal ein Beispiel an meiner Kousine Lucy nehmen. Lucy wäre ja so naturwissenschaftlich begabt. Der hat doch gar keine Ahnung! Mir wird schon schlecht, dass Lucy an Weihnachten wieder kommt.}
\milamx{Und dann ist morgen auch noch das Weihnachtskonzert von unserem Literaturkurs! Auf Weihnachten habe ich sowieso nie Lust. Alles ist stressig, und nervig, und ich bekomme eh immer nur kratzige Schals von Oma geschenkt. Aber was am schlimmsten ist, ist das ständige Generve von Mama, mit ihrer Fragerei, ob es mir gut geht und so. Egal, was ich mache, sie macht sich dauernd Sorgen, eine totale Helikoptermutter! Das nervt echt total!}
\milamx{\direct{seufzt} Aber naja, jetzt gehe ich erstmal schlafen, damit ich alles schaffe, und Mama nicht den Krankenwagen ruft, weil ich morgen so übermüdet aussehe. Auf die Krise kann ich gerne verzichten! Mal schauen, wie es morgen wird, ich werde auf jeden Fall wieder schreiben. Gute Nacht!}

\licht{Hintergrund wird dunkel}

\milamx{Achso, aber bevor ich es vergesse: Bei uns zu Hause gelten noch ein paar Regeln, damit niemand während unseres Zusammenlebens zu Schaden kommt! Vor allem gilt: Wir lassen unsere Handys während der Familienzeit ausgeschaltet. Da legt Papa sehr viel Wert drauf\dots Und im Notfall verlassen wir das Gebäude über die Notausgänge dort und dort \direct{zeigt Notausgänge}, das ist Mama sehr wichtig. \direct{genervt} Als ob jemals etwas passieren würde\dots \direct{stöhnt} Naja, was soll's, jetzt sollte ich wirklich schlafen.}

\mikroaus{1}
\StageDir{Licht aus.}