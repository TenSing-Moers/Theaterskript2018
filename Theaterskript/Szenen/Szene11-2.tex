% This file is part of TenSing-Moers/Theaterskript2018.
%
% TenSing-Moers/Theaterskript2018 is free content: you can redistribute and/or
% modify it under the terms of the cc-by-nc-sa (Creative Commons
% Attribution-NonCommercial-ShareAlike) as released by the
% Creative Commons organisation, version 4.0.
%
% TenSing-Moers/Theaterskript2018 is distributed in the hope that it will be useful,
% but without any warranty.
%
% You should have received a copy of the cc-by-nc-sa-license along
% with this copy of TenSing-Moers/Theaterkskript2018. If not, see
% <https://creativecommons.org/licenses/by-nc-sa/4.0/legalcode>.
%
% Copyright TenSing Moers and all whose work and <3 went in this project.
\section{Das Gegengift}
\DisplayPersons
%\chars{Helge Schneider}
\ort{Mitsibitsi-Elektro-Halle}
\requ{Tisch, Laborutensilien (Bechergläser, Standfüße, Reagenzgläser, Schutzbrille, farbige Flüssigkeiten), Laborkittel, Wäscheklammer}

\StageDir{\remyh steht an einem Tisch mit lauter Laborutensilien und mischt das Gegengift, \willyh, \doch, \schneewittchenh, \thomash, \pocahontash, \meridah und \jackh schauen ihm gespannt dabei zu. \winnieh sitzt etwas abseits weiterhin gefesselt auf seinem Stuhl, zerrt immer wieder an den Fesseln, und schafft es, sie nach und nach zu lösen}
\mikroan{1,2,3,4,5,6,7,8}

\begin{drama}
\schneewittchenx{\direct{trägt eine Wäscheklammer auf der Nase} Ist das ekelhafte Gebräu jetzt mal endlich fertig? Das stinkt einfach entsetzlich!}
\meridax{Dann geh doch woanders hin! Wenn du nicht an der Rettung unserer Welt interessiert bist, dann können wir dich hier auch nicht gebrauchen. \direct{Will \schneewittchen wegschieben}}
\docx{\direct{geht dazwischen} Wir müssen zusammenarbeiten!}
\thomasx{Was fehlt denn jetzt noch, \remy?}
\olafx{\direct{kommt fröhlich pfeifend auf die Bühne} Ohh, es riecht hier so nach Sommer! Riecht ihr die Blumen und den Sonnenschein?}
\pocahontasx{Seht ihr? Endlich jemand, der die Natur zu schätzen weiß!}
\schneewittchenx{\direct{angewidert} Natur? Sommer?! Also ich finde\dots}
\aladdinx{\direct{Kam mit \olaf auf die Bühne, setzt ihren Satz fort} \dots\ eigentlich stinkt es hier wie im Pumakäfig.}
\remyx{Ich bin ja auch schon fast fertig. Mir fehlt nur noch ein bisschen Spülmittel!}
\aladdinx{\direct{Greift nach dem Spülmittel, und reicht es \remy} Hier, ich hab Spülmittel!}
\remyx{Vielen Dank, ich brauche auch nur einen kleineeeen\dots}
\end{drama}

\StageDir{\remyh, der sich zu \aladdin gedreht hat, um das Spülmittel entgegen zu nehmen, dreht sich zurück zu seinem Tisch, und stößt dabei gegen Olaf, der umhergewuselt ist. Dabei drückt er aus Versehen eine große Menge Spülmittel aus der Flasche. Alle erschrecken und verschlagen die Hände über dem Kopf oder vor dem Mund. In dem Gewusel hat \winnieh es geschafft, sich zu befreien und verlässt die Bühne}

\begin{drama}
\remyx{\direct{resigniert} \dots\ Tropfen. \direct{verzweifelt} Oh nein oh nein, das war zu viel!}
\willyx{Das ist jetzt, gelinde gesagt, sehr schlecht. Das ganze Spülmittel ist in dem Gegengift!}
\meridax{Das hast du ja mal wieder toll hingekriegt, \aladdin.}
\aladdinx{Aber ich war das doch gar nicht! Es war \olaf!}
\olafx{\direct{schaut betreten zu Boden, traurig} 'Tschuldigung.}
\docx{Tja, dann müssen wir wohl nochmal von vorne anfangen.}
\schneewittchenx{Wie, noch länger hier rumstehen?! Ohne mich! \direct{Verlässt hochnäsig die Bühne}}
\remyx{Ich fürchte, das wird nicht möglich sein. Wir haben nicht mehr genügend Zutaten!}
\pocahontasx{So neumodisches chemisches Zeug ist sowieso schlecht für die Natur. Wenn wir doch nur alle auf den Wind hören und Mutter Erde schützen würden!}
\meridax{Ich bringe erstmal \olaf weg, bevor er noch mehr Schaden anrichtet. Komm, \olaf, wir gehen nach draußen!}
\end{drama}

\StageDir{\olafh läuft freudig zu \meridah, gemeinsam verlassen sie die Bühne}
\mikroaus{3,5}

\begin{drama}
\thomasx{Ich habe eine Idee! Kennt ihr Seifenblasen?}
\end{drama}

\StageDir{Alle schütteln den Kopf}

\begin{drama}
\thomasx{Da, wo ich herkomme, sind Seifenblasen sehr beliebt. Wenn genügend Spülmittel in einer Flüssigkeit ist, dann lassen sich daraus kleine Blasen formen, die durch die Luft schweben!}
\aladdinx{Ich verstehe nicht ganz\dots}
\thomasx{Auf diese Weise können wir uns dieses Missgeschick zu nutze machen! Wir verteilen unser Gegengift als Seifenblasen!}
\docx{Aber wie soll man denn aus einer Flüssigkeit Blasen formen?}
\thomasx{Bei uns nimmt man dazu einen Stab mit einem Ring dran.}
\aladdinx{So etwas wie das hier? \direct{hält einen Stab hoch, den er in diesem Moment aus Draht geformt hat}}
\thomasx{Exakt!}
\remyx{Das ist eine großartige Idee! Ich ergänze noch schnell die fehlenden Zutaten! \thomass, \aladdin, bastelt noch mehr von diesen Stäben. Und \doc, du kannst mir zur Hand gehen!}
\end{drama}

\mikroaus{1,2,4,6,7,8}
\StageDir{Licht aus.}