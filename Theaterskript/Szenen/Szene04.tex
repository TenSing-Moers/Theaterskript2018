% This file is part of TenSing-Moers/Theaterskript2018.
%
% TenSing-Moers/Theaterskript2018 is free content: you can redistribute and/or
% modify it under the terms of the cc-by-nc-sa (Creative Commons
% Attribution-NonCommercial-ShareAlike) as released by the
% Creative Commons organisation, version 4.0.
%
% TenSing-Moers/Theaterskript2018 is distributed in the hope that it will be useful,
% but without any warranty.
%
% You should have received a copy of the cc-by-nc-sa-license along
% with this copy of TenSing-Moers/Theaterkskript2018. If not, see
% <https://creativecommons.org/licenses/by-nc-sa/4.0/legalcode>.
%
% Copyright TenSing Moers and all whose work and <3 went in this project.
\section{Haltet die Diebe!}
\DisplayPersons
\DisplayRequisites
\ort{In einer Seitengasse in der Wutwelt}

\begin{drama}
\customx{Aus dem Off}{Haltet die Diebe!}
\end{drama}

\StageDir{Das Licht geht an. \miladh steht verwirrt auf der Bühne.}

\begin{drama}
\miladx{\direct{verwirrt} Hä? Was war das denn jetzt? Wo bin ich hier?!}
\albertx{\direct{begeistert} Ich fasse es nicht! Es hat tatsächlich gelappt! Ich bin ein Genie!}
\miladx{\direct{dreht sich zu \alberth um, genervt} Du bist ein Idiot. Nichts weiter. Warum hast du mich in die Maschine geschubst?!}
\albertx{\direct{triumphierend} Ach, stell dich doch nicht so an. Sei mir lieber dankbar, dass ich dich in eine andere Welt gebracht habe!}
\miladx{\direct{entgeistert} Bitte was?! Danken soll ich dir? Wie kann man denn so arrogant sein? Tu doch nicht so, als ob du in die Geschichtsbücher kommst!}
\end{drama}

\StageDir{\charlotteh, \sophiah, \emmah, \aaronh, \cleoh und \pikachuh jagen \wilmah und \wilhelminah von rechts nach links über die Bühne. Dabei fluchen sie:}

\begin{drama}
\customx{Meute}{Du Drecksdepp! Du hast mich bestohlen! Du Spachtelschwein! Du Käsekuchen! Ich weiß, wo dein Haus wohnt! Das wirst du bitter bezahlen! Und bereuen! Du Oberaffenschokokussvollpfosten! Vernichtung und ewiges Vergessen! Du Lauch! Du hast wohl nen Dachschaden! Retkahtaa! Kakkakikkare! Ristäjä! Din Dumbomm! Hur i helvete har hon hamnat här! Du, du, du\dots}
\end{drama}

\StageDir{\wilmah und \wilhelminah kommen von links wieder auf die Bühne, verstecken sich, die \customh{Meute} rennt an ihnen vorbei, und verlässt die Bühne wieder. \wilmah und \wilhelminah verlassen ihre Deckung, \wilmah telefoniert, \wilhelminah steht wütend daneben. \miladh und \alberth entdecken sie.}

\begin{drama}
\wilmax{\direct{ins Telefon} Wir brauchen HILFEEE!!}
\albertx{\direct{erstaunt} Ist das etwa ein Fernsprecher? Ohne Schnur? Ich bin fasziniert, das hätte ICH erfinden können! \direct{zückt einen \notizblockd und schreibt etwas auf}}
\miladx{\direct{schnippisch} Ehm ja, klar doch. Okay, das ist beeindruckend, das stimmt. Aber das ist doch der Dieb, den alle suchen, oder nicht?!}
\albertx{\direct{schnaubt verächtlich} Pf, das ist mir doch egal! Hallo?! Fernsprecher ohne Schnur?! \direct{Wendet sich an \wilmah} Ehm, Entschuldigung, dürfte ich mir das mal anschauen?}
\customx{\wilmah und \wilhelminah}{\direct{schreien} Nein, du NUDEL!}
\albertx{\direct{bettelt} Aber bitte, bitte!}
\miladx{\direct{entsetzt} \alberthh, bist du jetzt völlig durchgeknallt? Das ist ein Verbrecher! Und außerdem hat er nein gesagt. Also los, lass uns gehen!}
\albertx{\direct{arrogant} Ich werde mich nicht von deiner Ignoranz davon abhalten lassen, diese Welt hier zu erkunden!}
\wilhelminax{\direct{genervt} Sagt mal, könnt ihr vielleicht einfach mal\dots}
\wilmax{\direct{unterbricht}\dots\ die Schnauze halten?!}
\albertx{\direct{verwirrt} Was meinen Sie denn mit Schnauze? Der Begriff ist ja eigentlich gebräuchlich bei Schweinen... Ein sehr interessanter Sprachgebrauch! \direct{notiert sich wieder etwas in seinem \notizblockd}}
\customx{\charlotteh, \sophiah, \emmah, \aaronh, \cleoh und \pikachuh}{\direct{kommen wieder auf die Bühne} Da ist ja der Dieb!}
\end{drama}

\StageDir{Die Meute jagt \wilmah und \wilhelminah wieder von der Bühne. \alberth verfolgt den Pulk, \miladh bleibt alleine stehen.}

\begin{drama}
\miladx{\direct{zu sich selbst} Ich muss schon sagen, ich bin irgendwie beeindruckt. Dass sich hier alle gemeinsam gegen Verbrechen einsetzen! Und niemand stellt mir nervige Fragen oder gibt mir irgendwelche Aufgaben. Ich weiß zwar weder, wo ich hier bin, noch, wie ich hier hergekommen bin, aber ich kann mich ja genauso gut ein wenig umsehen!}
\end{drama}

\StageDir{\miladh verlässt die Bühne. Licht aus.}