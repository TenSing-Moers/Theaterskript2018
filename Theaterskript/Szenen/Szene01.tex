\Scene{Wetten, dass...?!}
\DisplayPersons
%\chars{Helge Schneider}
\ort{Wetten, dass...?!-Studio}
\requ{Eine Banane}

\StageDir{Wetten, dass...?! Intro mit Musik und Sprecher, eventuell als Videointro?}

\mikroan{2}
\sound{Moderator Intro}

\begin{drama}
\customx{Moderator}{Wetten, dass...?! - Live aus Moers! Der Rettungsschirm für den Samstagabend! Mit dabei im Spaßpaket: \ron mit einem unglaublichen Verschwindezauber, \kevin mit einhundert Waterbottle-Flips in nur einer Minute, \jacqueline, mit dem perfekten Make-up für drei Personen in nur zwei, \jeremy mit der Rückwärtsbuchstabierung der 10 längsten Wikipedia-Artikel weltweit, uuuuuund natürlich ihr krisensicherer Moderator, \thomas!}
\end{drama}

\sound{Moderator Intro}
\sound{Applaus}
\mikroaus{2}

\StageDir{\thomash betritt unter tosendem Applaus die Bühne.}
\mikroan{1}

\begin{drama}
\thomasx{Guten Abend! Herzlich Willkommen \krawatte zu \myquote{Wetten, dass...?!} Zunächst begrüßen wir einen, wie soll ich sagen \dots\ \krawatte etwas außergewöhnlichen Gast, mit einer noch außergewöhnlicheren Wette! Wenn ich ehrlich sein soll, habe ich diese Wette erst für einen Scherz gehalten, doch hier ist er: \ron!}
\end{drama}

\StageDir{Applaus, \ronh betritt die Bühne.}
\sound{Applaus}
\mikroan{4}

\begin{drama}
\thomasx{Setzen Sie sich doch zu mir, Herr \weasley. Darf ich sie \ronn nennen?}
\thomasx{Sie wollen also jemanden verschwinden lassen, \dots\ und wieder sicher ins Studio zurückholen, versteht sich. Was würden sie sagen, wie \dots\ ähh \dots\ sicher sind sie sich denn, dass sie diese Wette gewinnen werden?}
\ronx{\direct{unsicher} Nun ja \dots\ Eigentlich \dots\ ziemlich sicher. Ich hab das schon ein paar Mal gemacht, wissen Sie?}
\thomasx{Ja, die Jugend, immer so ehrgeizig und selbstsicher! Und das bringt mich auch gleich zu unserem nächsten Gast: Begrüßen sie den unglaublichen \kevin!}
\end{drama}

\StageDir{Applaus. \kevinh kommt selbstsicher mit einer Flasche auf die Bühne und macht es sich ungefragt auf dem Sofa bequem. \thomash sieht ihn an, und versucht, nicht genervt auszusehen.}
\mikroan{5}

\begin{drama}
\thomasx{\krawatte \kevin, Sie wetten, dass sie einhundert Waterbottle-Flips in einer Minute schaffen. Wie sicher sind Sie sich denn dabei?}
\kevinx{\direct{Hantiert wild mit seiner Flasche} Was soll das denn jetzt heißen, natürlich bin ich sicher! Sonst wär' ich ja nicht \direct{lässt aus Versehen die Flasche fallen, hebt sie aber sofort wieder auf, und redet weiter, als wäre nichts gewesen} hier! Und jetzt quatschen Sie keine Opern, und holen sie diese heiße Schnitte auf die Bühne, die mit der Schminke!}
\thomasx{\krawatte Äh \dots\ \direct{sichtlich schockiert} Ach, Sie meinen wohl \jacqueline, unsere nächste Kandidatin! Nun, natürlich will ich niemanden zu lange warten lassen: Hier kommt unsere dritte Wette, \jacqueline!}
\end{drama}

\StageDir{Applaus, \jacquelineh betritt die Bühne.}
\sound{Applaus}
\mikroan{6}

\begin{drama}
\thomasx{Setzen Sie sich doch, \jacqueline.}
\jacquelinex{Hatte ich auch vor, mein Goldlöckchen! \direct{setzt sich zu \kevin} Übrigens bräuchten Sie echt mal nen neuen Haarschnitt, und diese Krawatte...}
\thomasx{\krawatte}
\end{drama}

\StageDir{\thomash und \ronh sehen skeptisch zu den beiden herüber}

\begin{drama}
\thomasx{\jacqueline, sie sehen sich selbst als eine Art Expertin für Make-up \dots}
\jacquelinex{\direct{sauer} Jetzt hören Sie mir aber mal gut zu, Goldlöckchen: Ich halt' mich nicht für 'ne Expertin, ich bin eine! Ich bin ein Profi! Und jetzt machen Sie mal hinne, oder denken Sie, ich hab' Bock, hier ewig mit 'nem Wischmopp, 'nem Zirkusaffen und 'nem Freak mit Klamotten aus dem letzten Jahrhundert rumzuchillen?}
\ronx{\direct{Betrachtet skeptisch seine Kleidung} Die hab' ich aber erst seit gestern...}
\jacquelinex{Denkst du, das interessiert mich, Freakshow?}
\end{drama}

\StageDir{Es herrscht kurz Stille, dann ergreift \thomash das Wort}

\begin{drama}
\thomasx{\direct{Richtet gestresst seine Krawatte} Bevor wir zu den Wetten selbst kommen, muss ich natürlich noch unseren letzten Gast auf die Bühne bitten:}
\jacquelinex{\direct{Leise} Och nee...}
\thomasx{\jeremy!}
\jacquelinex{\direct{äfft ihn übertrieben sarkastisch nach}Järämih-Pascal}
\end{drama}

\StageDir{\jeremyh betritt zögerlich die Bühne, macht einen nervösen Eindruck}
\sound{Applaus}
\mikroan{3}

\begin{drama}
\thomasx{\direct{Beruhigend, freundlich} Kommen Sie, setzen Sie sich doch einfach hier zu mir!}
\end{drama}

\StageDir{\jeremyh setzt sich erleichtert zu \thomas und \ron}

\begin{drama}
\thomasx{\jeremy, Sie wetten, die zehn längsten Wikipedia-Artikel der Welt rückwärts buchstabieren zu können, und zwar auswendig. Wie fühlen Sie sich denn vor so einer Herausforderung?}
\jeremyx{\direct{nervös, stottert} Ich fühle mi-mich ga-ga-ganz gut, Herr \gottschalk. \direct{Sieht aber nicht danach aus} Ich habe mi-mich g-g-gründlich vorbereite, u-und gehe d-davon aus, dass ich e-es schaffen werde!}
\jacquelinex{Das kann doch jetzt nicht ihr ernst sein, Ringellöckchen! Ich sitze mir hier doch nicht den Hintern platt, nur damit dieser Vollfreak da den Weltrekord im ultralangsam Labern aufstellen kann!}
\jeremyx{Entschuldigen Sie bitte, aber meine Wette besteht darin, die zehn weltgrößten \dots}
\kevinx{\dots\ Wikipedia-Artikel rückwärts zu buchstabieren, schon klar! Träum weiter, du kannst ja nicht mal r-r-r-richtig s-sp-sprechen! \direct{äfft \jeremy nach.}}
\jeremyx{Äh, ich d-denke, Sie m-meinen Wikipedia, eine freie Online-Enzyklopädie, eines der vielen P-P-Projekte der Wi-Wi-Wikimedia Foundation, zusammen mit Wi-Wikiquote, Wiki-sp-species, Wiki-v-versity, \dots}
\kevinx{Glaubst du, das interessiert überhaupt nur eine Person, was du da laberst? Gut, dass ich vor dir dran bin, dann kriegen die ganzen Weiber hier wenigstens noch was zu sehen, bevor sie vor Langeweile ins Koma fallen. \direct{Wirft die Flasche}}
\jeremyx{Ähm, wenn Sie w-weniger W-Wasser verwenden w-würden, dann wäre der Schwerpunkt d-d-der Flasche deutlich w-w-weiter unten und sie stünde d-d-deutlich stabiler.}
\kevinx{Halt's Maul, ich brauch' mir von dir gar nichts erklären lassen! \direct{Sieht sich verstohlen um und trinkt dann trotzdem einen Schluck aus der Flasche}}
\ronx{Ja, äh, ich \dots\ wäre dann soweit \dots?}
\thomasx{Ja, lasst uns fortfahren.}
\end{drama}

\StageDir{\thomash und \ronh stehen auf, und stellen sich vorne an die Bühne}

\begin{drama}
\thomasx{Nun, meine Damen und Herren, machen Sie sich gefasst auf diese \dots\ spektakuläre Wette! Wir werden per Zufallsgenerator eine Person aus dem Publikum auswählen, die dann für uns \dots}
\ronx{\direct{unterbricht ihn} Herr \gottschalk, warum nehmen wir nicht einfach den/die da? \direct{deutet auf eine Person im Publikum}}
\thomasx{Nun, das können wir natürlich auch machen, ich liebe Spontanität, wissen Sie? \direct{Zu Person auf dem Publikum} Würden Sie bitte zu uns auf die Bühne treten?}
\end{drama}

\StageDir{\thomash stellt die \customh{Person aus dem Publikum} vorne auf die Bühne.}

\begin{drama}
\thomasx{Herr \weasley, Sie wetten, dass sie diese Person aus dem Studio verschwinden und wieder auftauchen lassen können. Sie dürfen dabei keinerlei Hilfsmittel verwenden, außer ihrem \dots}
\ronx{\direct{ergänzt} \dots\ außer meinem Zauberstab!}
\thomasx{\dots\ außer ihrem Zauberstab, genau. Gut, sind Sie bereit?}
\ronx{So bereit wie noch nie!}
\thomasx{Gut, dann: Topp, die Wette gilt!}
\end{drama}

\sound{Talkshow Intro}
\StageDir{Dramatische Melodie ertönt, Licht wird gedimmt, \customh{Spot} auf \ronn und die Person aus dem Publikum.}

\begin{drama}
\ronx{Bleiben Sie einfach ganz ruhig. Sie müssen jetzt an irgendeinen Ort denken, an dem Sie immer schon mal sein wollten. Es kann auch etwas verrücktes sein, nur bitte keine Märchenwelt, nichts mit Schlössern oder sprechenden Ratten, das wäre schlecht.}
\end{drama}

\mikroaus{1}
\StageDir{\ronh hebt seinen Zauberstab, Nebel, Stroboskop. \thomash verschwindet von der Bühne, während \ronh irgendetwas geheimnisvolles vor sich hinmurmelt. \soundwide{während er zaubert: Knall}Blitz und Knall, danach normales Theaterlicht}

\begin{drama}
\ronx{\direct{ruft} Ich hab's geschafft!}
\jacquelinex{Ey bist du dumm oder so? Die Person ist ja immer noch da, das sehe ja sogar ich! Von hier!}
\jeremyx{\thomass w-w-ird die S-Sache bestimmt gleich aufklären \dots\ T-T-Thomas?}
\kevinx{Leute, wo ist dieser \gottschalk hin?!}
\end{drama}

\mikroaus{3,4,5,6}
\StageDir{Licht aus.}