% This file is part of TenSing-Moers/Theaterskript2018.
%
% TenSing-Moers/Theaterskript2018 is free content: you can redistribute and/or
% modify it under the terms of the cc-by-nc-sa (Creative Commons
% Attribution-NonCommercial-ShareAlike) as released by the
% Creative Commons organisation, version 4.0.
%
% TenSing-Moers/Theaterskript2018 is distributed in the hope that it will be useful,
% but without any warranty.
%
% You should have received a copy of the cc-by-nc-sa-license along
% with this copy of TenSing-Moers/Theaterkskript2018. If not, see
% <https://creativecommons.org/licenses/by-nc-sa/4.0/legalcode>.
%
% Copyright TenSing Moers and all whose work and <3 went in this project.
\Scene{HARIBO\textsuperscript{\copyright \circledR} Goldbären\textsuperscript{TM \copyright \circledR}}
\DisplayPersons
%\chars{Helge Schneider}
\ort{Im Geheimversteck}
\requ{Seifenblasen, Seifenblasenmaschinen, Gummibärchen, Fesseln, Mario\textsuperscript{TM} Kart\textsuperscript{TM}-Startsignal}

\StageDir{\thomash, \willyh, \remyh, \olafh, \schneewittchenh, \doch, \aladdinh, \meridah und \jackh stehen aufbruchsbereit im Geheimversteck}
\mikroan{1,2,3,4,5,6}

\begin{drama}
\remyx{So, haben wir alles? Ich glaube, wir können los!}
\thomasx{Okay, hat jeder etwas von dem Gegengift? Denkt alle an den Plan \direct{schaut \olaf streng an} und macht alles genau so, wie wir es besprochen haben!}
\olafx{\direct{stolz} Ich hab mir alles gemerkt! Wir müssen das Zeugs\dots\ Ähh, das Gegengift auf die Leute draufspucken, damit sie Blasen bekommen!}
\docx{\direct{schlägt sich mit der Hand vor die Stirn} Pusten, nicht spucken! Und nicht die Leute sollen Blasen kriegen, sondern das Gegenmittel schwebt als Blasen durch die Luft! \direct{Macht es \olaf vor}}
\willyx{\direct{ermahnend} Verschwende das Mittel nicht, \doc}
\jackx{Also, ich möchte ja nicht drängeln, aber wir sollten uns ein bisschen beeilen!}
\schneewittchenx{Los jetzt, ich habe heute Abend auch noch was anderes, vor als diese Seifendingsda zu pusten!}
\thomasx{Also gut, folgt mir!}
\end{drama}

\StageDir{\thomash will von der Bühne gehen, die anderen folgen ihm. Plötzlich taucht die \rkh auf und stellt sich ihnen in den Weg}

\begin{drama}
\rkx{Ja, das habt ihr euch wohl so gedacht, hm?}
\end{drama}

\StageDir{Hinter der roten Königin kommen \mickeyh und ein paar \customh{Wachen} hervor}

\begin{drama}
\mickeyx{Ich werde nicht zulassen, dass ihr die Hypnose aufhebt!}
\meridax{Oh, wie schade nur, dass wir dich nicht um Erlaubnis gefragt haben!}
\end{drama}

\StageDir{\mickeyh sieht unsicher zur Königin}

\begin{drama}
\rkx{Ihr kommt an uns nicht vorbei! Ab mit ihren Köpfen!}
\olafx{Meinen Kopf kriegt ihr nicht!}
\aladdinx{Also gut, wenn ihr Kämpfen wollt, kämpfen wir!}
\end{drama}

\StageDir{Die \rkh und \mickey stellen sich an den Rand und beobachten den Kampf, in dem alle anderen aufeinander losgehen. Die Guten sind im Vorteil, da sie in der Überzahl sind.}

\begin{drama}
\pocahontasx{\direct{ruft} Stopp! Aufhören! Wohin soll das denn noch führen?!}
\rkx{Das Mädel hat recht. Das ist kein Kampf zwischen allen, es geht hier nur um \mickeyy und den Fremden da!}
\rkx{\mickeyy, fordere ihn zum Einzelduell!}
\mickeyx{Aber\dots\ Warum?!}
\rkx{Mach schon, \mickeyy! Wenn er gewinnt, \direct{deutet auf Thomas} lassen wir euch ziehen. Aber wenn \mickey gewinnt, seid ihr unsere Gefangenen!}
\end{drama}

\StageDir{Die Guten stecken die Köpfe zusammen und tuscheln}

\begin{drama}
\thomasx{Gut, ich nehme die Herausforderung an! Beenden wir es ein für alle mal!}
\end{drama}

\StageDir{Alle stellen sich im Halbkreis auf, \mickeyh und \thomash stehen sich in der Mitte gegenüber. \sound{Mario Kart} Es ertönt das Mario\textsuperscript{TM} Kart\textsuperscript{TM}-Startsignal, \mickeyh und \thomash fangen an zu kämpfen}

\begin{drama}
\thomasx{Ich weiß, dass du das nicht willst, \mickeyy. Die \rk benutzt dich, um selbst an die Macht zu kommen! Du musst das nicht tun!}
\end{drama}

\StageDir{Doch \mickeyh geht nicht darauf ein, sondern drängt \thomas nach und nach nach hinten. In der Not greift \thomas in seine Anzugtasche und fördert eine Tüte original beste erlesene Premium-HARIBO\textsuperscript{\copyright \circledR} Goldbären\textsuperscript{TM \copyright \circledR} Qualität 1a+ \textit{(gibt's auch hier im Kino)} zu Tage, er fängt wild an, mit original besten erlesenen Premium-HARIBO\textsuperscript{\copyright \circledR} Goldbären\textsuperscript{TM \copyright \circledR} Qualität 1a+ \textit{(gibt's auch hier im Kino)} zu werfen, \mickeyh geht zu Boden.}

\begin{figure}[t]
\centering
+++ Dies ist eine Dauerwerbeseite +++ Dies ist eine Dauerwerbeseite +++ Dies ist eine Dauerwerbeseite +++ Dies ist eine Dauerwerbeseite +++
\end{figure}

\begin{drama}
\thomasx{\direct{singt} HARIBO\textsuperscript{TM \copyright \circledR} macht Kinder froh, doch den \mickeyy gar nicht so!}
\jackx{Ich würde sagen, das war\dots\ interessant!}
\end{drama}

\StageDir{Die Guten gehen sofort auf die Bösen los, ringen sie zu Boden und fesseln sie.}

\begin{drama}
\rkx{Ihr Verräter! Ihr dürft mich nicht anrühren, ich bin die Königin! Ab mit ihren Köpfen!}
\end{drama}

\StageDir{\rkh wird geknebelt}

\begin{drama}
\olafx{\direct{zu \thomas} Du hast es geschafft! Ich bin so froh! Das ist der schönste Tag meines Lebens!}
\schneewittchenx{Ich bin so froh, das alles vorbei ist. Vor lauter Stress habe ich bestimmt schon graue Haare bekommen! Und ein Fingernagel ist abgebrochen, seht ihr? \direct{hält ihren Finger in die Luft}}
\remyx{Wie auch immer, jetzt sollten wir aber mal wirklich das Gegengift verteilen!}
\docx{Stimmt, aber schnell! Kommt, Freunde, los geht's!}
\end{drama}

\mikroaus{1,2,3,4,5,6}
\StageDir{Licht aus.}
