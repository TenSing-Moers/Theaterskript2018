% This file is part of TenSing-Moers/Theaterskript2018.
%
% TenSing-Moers/Theaterskript2018 is free content: you can redistribute and/or
% modify it under the terms of the cc-by-nc-sa (Creative Commons
% Attribution-NonCommercial-ShareAlike) as released by the
% Creative Commons organisation, version 4.0.
%
% TenSing-Moers/Theaterskript2018 is distributed in the hope that it will be useful,
% but without any warranty.
%
% You should have received a copy of the cc-by-nc-sa-license along
% with this copy of TenSing-Moers/Theaterkskript2018. If not, see
% <https://creativecommons.org/licenses/by-nc-sa/4.0/legalcode>.
%
% Copyright TenSing Moers and all whose work and <3 went in this project.
\section{437,1 Minuten}
\DisplayPersons
\DisplayRequisites
\ort{In der hektischen Welt}

\StageDir{Die Szene beginnt mit einem \weckerklingelnd. \miladh, \alberth, \wilmah, \wilhelminah, \cosmah und \salomeh sitzen weiterhin auf dem Boden. Um sie laufen hektisch Leute herum und telefonieren, gestikulieren wild, schauen in ihre Terminkalender.}

\begin{drama}
\miladx{\direct{steht auf, schaut sich um, verwundert, überrascht} Wow, hier sind aber viele Leute! Und alle sehen so aus, als hätten sie was zu tun. Nicht so wie gerade, wo alle keine Hobbys hatten!}
\end{drama}

\StageDir{Die anderen stehen ebenfalls auf.}

\begin{drama}
\wilmax{Ernsthaft? Noch 'ne dritte Welt? Och nee\dots}
\wilhelminax{Langsam reichts mir! Ich will euch nicht mehr länger ertragen, ihr \textbf{NUDELN!}}
\salomex{Ach, chill mal, \wilmahh. Wir machen uns einfach ne eeeeasy time hier, ist doch völlig wayne, wo wir sind, chillen kann man schließlich überall!}
\albertx{Und ich bin wieder höchst beeindruckt! Mit was für Mitteln man durch Welten reisen kann\dots Da eröffnen sich mir als Wissenschaftler völlig neue Welten!}
\end{drama}

\mycomment{Tut mir Leid, aber der musste sein, schlagt mich bitte nicht!}

\begin{drama}
\cosmax{Ach, das freut mich aber\dots Aber neeein!!}
\salomex{\direct{genervt} Was ist denn los? Bleib doch mal locker!}
\cosmax{\direct{verzweifelt} Irgendwie ist mein Kuchen nicht mehr da! Ich hab doch extra ganz viel gemacht, damit es für mehrere Welten ausreicht\dots}
\albertx{\direct{neunmalklug} Aber du hast den doch auf den Boden gestellt! Natürlich kommt der dann nicht mit\dots}
\cosmax{Och mist, neeein! \direct{schlägt sich vor die Stirn}}
\wilmax{Hahaha, wie war das nochmal?}
\wilhelminax{\direct{äfft sie nach} Ich will ja nicht so enden wie \wilmah und \wilhelminah! Bla Bla Bla!}
\cosmax{\direct{In sich hinein} Einmal Liebe, zweimal Liebe, dreimal Liebe\dots}
\miladx{Ach, hört doch auf, euch zu Streiten. Ich will lieber was erleben!}
\salomex{Schon wieder so motiviert! Lasst uns doch lieber einen groovy Spot zum chillen suchen!}
\miladx{Du kannst dir gerne deinen \glqq groovy Spot\grqq \direct{ahmt sie nach} suchen, aber mir gefällt der Tatendrang der Leute hier!}
\miladx{\direct{Wendet sich an \theoh und \cleoh, die vorbeilaufen} Hey, wisst ihr, wo man hier was erleben kann?}
\end{drama}

\StageDir{\theoh's Handywecker klingelt (\weckerklingelnd). Er holt sofort hektisch sein Handy aus der Tasche und schaut darauf.}

\begin{drama}
\theox{\direct{gehetzt} Oh, entschuldige. Ich habe nur noch 3,21 Minuten Zeit, um zu meiner Bushaltestelle zu laufen. Aber wenn du Lust hast, kannst du gerne mit meiner Tochter Kunigunde shoppen gehen! Das müsste doch noch in deinen Terminkalender passen, oder, Liebling?}
\cleox{\direct{freudig} Super, dann muss ich nicht alleine gehen! Ich hätte noch 37,95 Minuten Zeit für dich, danach habe ich 47,37 Minuten lang Reitstunde, 17,23 Minuten Fechttraining und muss noch genau 32,71 Minuten Hausaufgaben machen, damit ich nach 12,68 Minuten Essen und Bettfertigmachen 473,1 Minuten schlafen kann!}
\salomex{\direct{entsetzt} Oh Gott, mir wird kotzübel. Was habt'n ihr so'n Stress?!}
\miladx{\direct{etwas überwältigt von dem Angebot, aber hält es für eine nette Geste} Ehm... Na klar, das wäre sehr cool! Wann denn?}
\cleox{\direct{verwundert} Wie, wann denn? Na, jetzt sofort natürlich! Sonst passt meine Zeitplanung doch gar nicht mehr! Und dein Terminkalender ist doch sicherlich auch ganz schön voll, oder? Aber lass deine komischen Freunde bitte hier, sonst dauert das alles zu lange.}
\wilmax{\direct{empört} Wie bitte?!}
\wilhelminax{Freunde?!}
\miladx{Ähm\dots\ okay!}
\end{drama}

\StageDir{\cleoh zieht \miladh hinter sich her von der Bühne. \theoh verlässt die Bühne in die andere Richtung. Der Rest bleibt verwundert stehen. Spot auf \milamh an der Bühnenkante:}

\begin{drama}
\milamx{Liebes Tagebuch, das shoppen gehen mit Kunigunde war echt schön, ich habe ein paar coole Sachen gesehen. Aber ich konnte mir nichts kaufen, denn Kunigunde war viel zu hektisch und wollte immer sofort weitergehen. Irgendwie war es auch echt anstrengend, weil sie dauernd auf die Uhr geguckt hätte, als hätte sie was zu verlieren. Sie hat sogar ausgerechnet, wie viel Zeit wir in den einzelnen Läden haben! Total verrückt. Alles war mega durchgetaktet, und sie hat mich auch ständig nach meinem Terminkalender gefragt. Mir ist dann bewusst geworden, dass auch dieser Ort nicht das ist, was ich eigentlich möchte. Ich habe sogar \cosmahh, \wilmahh, \wilhelminahh, \salomehh und \alberthh ein bisschen vermisst.}
\milamx{Die ganzen Reisen haben mir gezeigt, dass nicht alles so gut ist, wie es scheint. An jedem Ort dachte ich erst, dass es perfekt ist. Aber eigentlich war es jedes Mal sogar noch schlimmer als vorher! Vermutlich ist es wirklich am besten, eine Mischung aus allem zu haben. Weder hasserfüllt, noch überfürsorglich. Eigentlich genau wie Mama! Nicht zu gleichgültig, aber auch nicht zu hektisch, so wie \leonhh und \lenahh. Eigentlich nerven die ja doch gar nicht so dolle, ich vermisse sie sogar. Und auch mit meiner Kusine Lucy\dots\ Ich war nie wirklich nett zu ihr, vielleicht sollte ich ihr auch nochmal eine Chance geben.}
\milamx{Wenn ich so darüber nachdenke, hatte ich eigentlich genau das, was ich wollte. Klar bedeutet Weihnachten immer Stress, aber am Ende verbringt man eine schöne Zeit mit der Familie. Ich wünschte, ich wäre einfach wieder zu Hause.}
\end{drama}

\StageDir{Licht aus.}