% This file is part of TenSing-Moers/Theaterskript2018.
%
% TenSing-Moers/Theaterskript2018 is free content: you can redistribute and/or
% modify it under the terms of the cc-by-nc-sa (Creative Commons
% Attribution-NonCommercial-ShareAlike) as released by the
% Creative Commons organisation, version 4.0.
%
% TenSing-Moers/Theaterskript2018 is distributed in the hope that it will be useful,
% but without any warranty.
%
% You should have received a copy of the cc-by-nc-sa-license along
% with this copy of TenSing-Moers/Theaterkskript2018. If not, see
% <https://creativecommons.org/licenses/by-nc-sa/4.0/legalcode>.
%
% Copyright TenSing Moers and all whose work and <3 went in this project.
% After some kind answer on http://tex.stackexchange.com/questions/357020/latex-output-custom-commands-used-in-current-section

\usepackage{refcount}
\usepackage{xparse}

% This file is part of TenSing-Moers/Theaterskript2018.
%
% TenSing-Moers/Theaterskript2018 is free content: you can redistribute and/or
% modify it under the terms of the cc-by-nc-sa (Creative Commons
% Attribution-NonCommercial-ShareAlike) as released by the
% Creative Commons organisation, version 4.0.
%
% TenSing-Moers/Theaterskript2018 is distributed in the hope that it will be useful,
% but without any warranty.
%
% You should have received a copy of the cc-by-nc-sa-license along
% with this copy of TenSing-Moers/Theaterkskript2018. If not, see
% <https://creativecommons.org/licenses/by-nc-sa/4.0/legalcode>.
%
% Copyright TenSing Moers and all whose work and <3 went in this project.

%Zwei Farben, ein Grauton und Schwarz
\definecolor{grey}{RGB}{100, 100, 100}
\definecolor{black}{RGB}{0, 0, 0}

% Neues Farbschema, basierend auf Regenbogen-Farbschema
\definecolor{colora}{RGB}{252,234,16}
\definecolor{colorb}{RGB}{249,178,51}
\definecolor{colorc}{RGB}{243,146,0}
\definecolor{colord}{RGB}{233,78,27}
\definecolor{colore}{RGB}{227,6,19}
\definecolor{colorf}{RGB}{163,25,91}
\definecolor{colorg}{RGB}{230,0,126}
\colorlet{colorh}{DeepPink1}
\colorlet{colori}{Magenta1}
\colorlet{colorj}{Magenta3}

\definecolor{colork}{RGB}{149,27,129}
\definecolor{colorl}{RGB}{102,36,131}
\definecolor{colorm}{RGB}{41,35,92}
\definecolor{colorn}{RGB}{45,46,131}
\definecolor{coloro}{RGB}{29,113,184}
\definecolor{colorp}{RGB}{54,169,225}
\definecolor{colorq}{RGB}{0,161,154}
\definecolor{colorr}{RGB}{47,172,102}
\definecolor{colors}{RGB}{0,102,51}
\definecolor{colort}{RGB}{58,170,53}
\definecolor{coloru}{RGB}{149,193,31}
\definecolor{colorv}{RGB}{177,127,74}
\definecolor{colorw}{RGB}{125,78,36}
\definecolor{colorx}{RGB}{67,41,24}
\colorlet{colory}{Ivory2}
\colorlet{colorz}{Ivory4}

% Charaktere: Wetten, dass...?!
\colorlet{thomasc}{colora}
\colorlet{jeremyc}{colorb}
\colorlet{ronc}{colorc}
\colorlet{kevinc}{colord}
\colorlet{jacquelinec}{colore}
\colorlet{wacheac}{colorf}
\colorlet{wachebc}{colorg}
% Charaktere: Die Guten
\colorlet{olafc}{colorh}
\colorlet{meridac}{colori}
\colorlet{aladdinc}{colorj}
\colorlet{jackc}{colork}
\colorlet{remyc}{colorl}
\colorlet{pocahontasc}{colorm}
\colorlet{moderatorc}{colorn}
% Charaktere: Die Bösen
\colorlet{mickeyc}{coloro}
\colorlet{schneewittchenc}{colorp}
\colorlet{docc}{colorq}
\colorlet{rkc}{colorr}
\colorlet{maleficentc}{colors}
\colorlet{willyc}{colort}
\colorlet{winniec}{coloru}
% Charaktere: Die Manager
\colorlet{managerac}{colorv}
\colorlet{managerbc}{colorw}
\colorlet{managercc}{colorx}
\colorlet{managerdc}{colory}
\colorlet{managerec}{colorz}

%%%%%%%%%%%%%%%%%%%%%%%%%%%%%%%%%%%%%%%%%%%
%%%%%%%%%%%% Alte Farbschemata %%%%%%%%%%%%
%%%%%%%%%%%%%%%%%%%%%%%%%%%%%%%%%%%%%%%%%%%

% Regenbogen-Farbschema, Farben aus Adobe Illustrator CC 2015, Namen aus TEN SING Theaterstück 2017
%\definecolor{colora}{RGB}{252,234,16}
%\definecolor{colorb}{RGB}{249,178,51}
%\definecolor{colorc}{RGB}{243,146,0}
%\definecolor{colord}{RGB}{233,78,27}
%\definecolor{colore}{RGB}{227,6,19}
%\definecolor{colorf}{RGB}{230,0,126}
%\definecolor{colorg}{RGB}{214,11,82}
%\definecolor{colorh}{RGB}{163,25,91}
%\definecolor{colori}{RGB}{149,27,129}
%\definecolor{colorj}{RGB}{102,36,131}
%\definecolor{colork}{RGB}{41,35,92}
%\definecolor{colorl}{RGB}{45,46,131}
%\definecolor{colorm}{RGB}{29,113,184}
%\definecolor{colorn}{RGB}{54,169,225}
%\definecolor{coloro}{RGB}{47,172,102}
%\definecolor{colorp}{RGB}{0,102,51}
%\definecolor{colorq}{RGB}{58,170,53}
%\definecolor{colorr}{RGB}{149,193,31}
%\definecolor{colors}{RGB}{177,127,74}
%\definecolor{colort}{RGB}{125,78,36}
%\definecolor{coloru}{RGB}{67,41,24}
%\definecolor{colorv}{RGB}{0,161,154}
%\definecolor{colorw}{RGB}{58,170,53}
%\definecolor{colorx}{RGB}{203,187,160}
%\definecolor{colory}{RGB}{203,187,160}
%\definecolor{colorz}{RGB}{203,187,160}

%New Colors generated with my rainbow color generator for Theaterskript 2017:
%see res/RainbowGenerator/rainbowgenerator.c
%\definecolor{colora}{RGB}{191,063,063}
%\definecolor{colorb}{RGB}{191,034,063}
%\definecolor{colorc}{RGB}{191,004,063}
%\definecolor{colord}{RGB}{191,230,063}
%\definecolor{colore}{RGB}{191,201,063}
%\definecolor{colorf}{RGB}{210,191,063}
%\definecolor{colorg}{RGB}{240,191,063}
%\definecolor{colorh}{RGB}{014,191,063}
%\definecolor{colori}{RGB}{044,191,063}
%\definecolor{colorj}{RGB}{063,191,053}
%\definecolor{colork}{RGB}{063,191,024}
%\definecolor{colorl}{RGB}{063,191,250}
%\definecolor{colorm}{RGB}{063,191,220}
%\definecolor{colorn}{RGB}{063,191,191}
%\definecolor{coloro}{RGB}{063,220,191}
%\definecolor{colorp}{RGB}{063,250,191}
%\definecolor{colorq}{RGB}{063,024,191}
%\definecolor{colorr}{RGB}{063,053,191}
%\definecolor{colors}{RGB}{044,063,191}
%\definecolor{colort}{RGB}{014,063,191}
%\definecolor{coloru}{RGB}{240,063,191}
%\definecolor{colorv}{RGB}{210,063,191}
%\definecolor{colorw}{RGB}{191,063,201}
%\definecolor{colorx}{RGB}{191,063,230}
%\definecolor{colory}{RGB}{191,063,004}
%\definecolor{colorz}{RGB}{191,063,034}

%New Color theme made of one color of each x11names-color-set
%%%%%%%%%%%%%%%%%%%%%%%%%%%%%%%%%
%%%% Colors to choose from: %%%%%
%%%%%%%%%%%%%%%%%%%%%%%%%%%%%%%%%
% One of each:
% - Beige: AntiqueWhite, Bisque, Burlywood, Cornsilk, LemonChiffon, LightYellow, MistyRose, NavajoWhite, PeachPuff, [[Wheat(1)]]
% - Türkis: [[Aquamarine(1)]], Turquoise
% - Grau: Azure, Honeydew, Ivory, LavenderBlush, Seashell, [[SlateGray(1)]], Snow,
% - Dunkelblau: Blue, DeepSkyBlue, DodgerBlue, RoyalBlue, SkyBlue, SteelBlue
% - Rot: Brown, Firebrick, IndianRed, OrangeRed, Red, Tomato
% - Hellblau: CadetBlue, Cyan, DarkSlateGray, DeepSkyBlue, LightBlue, LightCyan, LightSkyBlue, LightSteelBlue, PaleTurquoise, SkyBlue, SteelBlue
% - Hellgrün: Chartreuse, DarkOliveGreen, DarkSeaGreen, Green, OliveDrab, PaleGreen, SeaGreen, SpringGreen
% - Dunkelgrün: Chartreuse, DarkOliveGreen, DarkSeaGreen, Green, OliveDrab, PaleGreen, SeaGreen, SpringGreen
% - Orange: Chocolate, DarkOrange, Coral, LightSalmon, Orange, Sienna
% - Braun: Chocolate, Salmon, Sienna, Tan
% - Gold: DarkGoldenrot, Gold, Goldenrot, LightGoldenrot
% - Lila: DarkOrchid, Magenta, Maroon, MediumOrchid, MediumPurple, Orchid, Plum, Purple, SlateBlue
% - Pink: DeepPink, HotPink, LightPink, PaleVioletRed, VioletRed
% - Gelb: Khaki, Yellow
% - Rosa: Pink, RosyBown, PaleVioletRed, Thistle

%\colorlet{colora}{AntiqueWhite1}
%\colorlet{colorb}{Aquamarine1}
%\colorlet{colorc}{Azure1}
%\colorlet{colord}{Bisque1} %color set "Blue" skipped
%\colorlet{colore}{Brown1}
%%color set "Burlywood" skipped (similarity to "AntiqueWhite)
%\colorlet{colorf}{CadetBlue2}
%\colorlet{Chartreuse3}
% This file is part of TenSing-Moers/Theaterskript2018.
%
% TenSing-Moers/Theaterskript2018 is free content: you can redistribute and/or
% modify it under the terms of the cc-by-nc-sa (Creative Commons
% Attribution-NonCommercial-ShareAlike) as released by the
% Creative Commons organisation, version 4.0.
%
% TenSing-Moers/Theaterskript2018 is distributed in the hope that it will be useful,
% but without any warranty.
%
% You should have received a copy of the cc-by-nc-sa-license along
% with this copy of TenSing-Moers/Theaterkskript2018. If not, see
% <https://creativecommons.org/licenses/by-nc-sa/4.0/legalcode>.
%
% Copyright TenSing Moers and all whose work and <3 went in this project.

\usepackage{xparse}
\ExplSyntaxOn
\NewDocumentCommand{\newnumberedlist}{ m }
 {
  \seq_new:c { g_numbered_#1_seq }
 }
\NewDocumentCommand{\addtonumberedlist}{ O{numberedlist} m }
 {
  \seq_gput_right:cn { g_numbered_#1_seq } { #2 }
 }
\NewExpandableDocumentCommand{\getnthelement}{ O{numberedlist} m }
 {
  \seq_item:cn { g_numbered_#1_seq } { #2 }
 }
\NewDocumentCommand{\storenthelement}{ O{numberedlist} m m }
 {
  \cs_set:Npx #3 { \seq_item:cn { g_numbered_#1_seq } { #2 } }
 }
\NewDocumentCommand{\clearnumberedlist}{ O{numberedlist} }
 {
  \seq_gclear:c { g_numbered_#1_seq }
 }
\ExplSyntaxOff

\newnumberedlist{colors}
\addtonumberedlist{colora}
\addtonumberedlist{colorb}
\addtonumberedlist{colorc}
\addtonumberedlist{colord}
\addtonumberedlist{colore}
\addtonumberedlist{colorf}
\addtonumberedlist{colorg}
\addtonumberedlist{colorh}
\addtonumberedlist{colori}
\addtonumberedlist{colorj}
\addtonumberedlist{colork}
\addtonumberedlist{colorl}
\addtonumberedlist{colorm}
\addtonumberedlist{colorn}
\addtonumberedlist{coloro}
\addtonumberedlist{colorp}
\addtonumberedlist{colorq}
\addtonumberedlist{colorr}
\addtonumberedlist{colors}
\addtonumberedlist{colort}
\addtonumberedlist{coloru}
\addtonumberedlist{colorv}
\addtonumberedlist{colorw}
\addtonumberedlist{colorx}
\addtonumberedlist{colory}
\addtonumberedlist{colorz}

% number of the next color to use when none is specified.
\newcounter{colornumber}
\stepcounter{colornumber}

\ExplSyntaxOn
\seq_new:N \g_luke_listofpersons_seq
\seq_new:N \l_luke_listofpersons_seq 


\NewDocumentCommand{\addperson}{m}{%
  \seq_gput_right:Nn \g_luke_listofpersons_seq {#1}
  \seq_gremove_duplicates:N \g_luke_listofpersons_seq
}


\NewDocumentCommand{\addpersonlocal}{m}{%
  \seq_gput_right:Nn \l_luke_listofpersons_seq {#1}
}

\cs_new:Npn \IfPersonCalledAlreadyF #1#2 {%
  \seq_if_in:NnF \l_luke_listofpersons_seq {#1} {#2}
}

\NewDocumentCommand{\DisplayPersons}{}{%
  \seq_clear:N \l_luke_listofpersons_seq
  \group_begin:
  \seq_clear:N \l_tmpa_seq
  \seq_map_inline:Nn \g_luke_listofpersons_seq {%
    \IfRefUndefinedExpandable{##1\thesection}{}{
      \seq_put_right:Nn \l_tmpa_seq {\use:c{##1h}}
     }
   }
   \seq_if_empty:NF \l_tmpa_seq {
     \PrePersonList
     \seq_use:Nn \l_tmpa_seq {,~} % this is the delimiter
     \PostPersonList
   }
   \group_end:

}
\ExplSyntaxOff

\NewDocumentCommand{\PostPersonList}{}{%
  \bigskip%

}

\NewDocumentCommand{\displayindividualperson}{m}{%
  \textbf{#1}%
}

\NewDocumentCommand{\PrePersonList}{}{%
  {\large \bfseries Charaktere:}%
}

% \NewPerson: Creates a new Person with its respective commands
% First Argument:  The Person's full name (as shown in the document)
% Second Argument: The Person's short name (used for the command creation)
% Third Argument:  The Person's highlight color (used for highlighting and 
%                  colored boxes).
% Fourth Argument: The Person's text color (used for displaying the person's
%                  name on the colored background).
\makeatletter
\NewDocumentCommand{\NewPerson}{m+m+m+m}{%
  \typeout{Creating person #1...}
  % Add this person to the global list
  \addperson{#2}%
  % Now define the personal \...x command 
  \expandafter\NewDocumentCommand\csname #2\endcsname{}{%
    \textsc{#1} %
  }
  \expandafter\NewDocumentCommand\csname #2x\endcsname{+m}{%
    %Check if the person has been called in the local section already
    \IfPersonCalledAlreadyF{#2}{%
      \addpersonlocal{#2}
      % Add the personal to the local list, i.e. per section
      % Check whether the label has been defined already
        \protected@edef\@currentlabel{\thesection.#2}\label{#2\thesection}
    }%
    \speechbox{#1}{#3}{#4}{##1}%
  } % defines \<charactername>x-command for speech
  \expandafter\NewDocumentCommand\csname #2h\endcsname{}{%
  %Check if the person has been called in the local section already
    \IfPersonCalledAlreadyF{#2}{%
      \addpersonlocal{#2}
      % Add the personal to the local list, i.e. per section
      % Check whether the label has been defined already
        \protected@edef\@currentlabel{\thesection.#2}\label{#2\thesection}
    }%
    \boxcommand{#3}{\strut\textcolor{#4}{\textsc{#1}}}
  } % defines \<charactername>h-command for inline highlighting
}% End of \NewPerson
\makeatother

% \NewPerson: Creates a new Person with its respective commands. The colors
% are taken from the 26 colors in the highlightcolors- and textcolors-lists.
% First Argument:  The Person's full name (as shown in the document)
% Second Argument: The Person's short name (used for the command creation)
\makeatletter
\NewDocumentCommand{\NewPersonAutoColor}{m+m}{%
  \typeout{Creating Person #1 with automatic colors...}
  \edef\currentnumber{\value{colornumber}}
  % Add this person to the global list
  \addperson{#2}%
  % Now define the personal \...x command 
  \expandafter\NewDocumentCommand\csname #2\endcsname{}{%
    \textsc{#1} %
  }
  \expandafter\NewDocumentCommand\csname #2x\endcsname{+m}{%
    %Check if the person has been called in the local section already
    \IfPersonCalledAlreadyF{#2}{%
      \addpersonlocal{#2}
      % Add the personal to the local list, i.e. per section
      % Check whether the label has been defined already
        \protected@edef\@currentlabel{\thesection.#2}\label{#2\thesection}
    }%
    \speechbox{#1}{\getnthelement[highlightcolors]{\currentnumber}}{\getnthelement[textcolors]{\currentnumber}}{##1}%
  } % defines \<charactername>x-command for speech
  \expandafter\NewDocumentCommand\csname #2h\endcsname{}{%
  %Check if the person has been called in the local section already
    \IfPersonCalledAlreadyF{#2}{%
      \addpersonlocal{#2}
      % Add the personal to the local list, i.e. per section
      % Check whether the label has been defined already
        \protected@edef\@currentlabel{\thesection.#2}\label{#2\thesection}
    }%
    \boxcommand{\getnthelement[highlightcolors]{\currentnumber}}{\strut\textcolor{\getnthelement[textcolors]{\currentnumber}}{\textsc{#1}}}
  } % defines \<charactername>h-command for inline highlighting
\stepcounter{colornumber}
}% End of \NewPerson
\makeatother