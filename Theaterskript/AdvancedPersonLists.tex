% This file is part of TenSing-Moers/Theaterskript2018.
%
% TenSing-Moers/Theaterskript2018 is free content: you can redistribute and/or
% modify it under the terms of the cc-by-nc-sa (Creative Commons
% Attribution-NonCommercial-ShareAlike) as released by the
% Creative Commons organisation, version 4.0.
%
% TenSing-Moers/Theaterskript2018 is distributed in the hope that it will be useful,
% but without any warranty.
%
% You should have received a copy of the cc-by-nc-sa-license along
% with this copy of TenSing-Moers/Theaterkskript2018. If not, see
% <https://creativecommons.org/licenses/by-nc-sa/4.0/legalcode>.
%
% Copyright TenSing Moers and all whose work and <3 went in this project.
% After some kind answer on http://tex.stackexchange.com/questions/357020/latex-output-custom-commands-used-in-current-section

\usepackage{refcount}
\usepackage{xparse}

% This file is part of TenSing-Moers/Theaterskript2018.
%
% TenSing-Moers/Theaterskript2018 is free content: you can redistribute and/or
% modify it under the terms of the cc-by-nc-sa (Creative Commons
% Attribution-NonCommercial-ShareAlike) as released by the
% Creative Commons organisation, version 4.0.
%
% TenSing-Moers/Theaterskript2018 is distributed in the hope that it will be useful,
% but without any warranty.
%
% You should have received a copy of the cc-by-nc-sa-license along
% with this copy of TenSing-Moers/Theaterkskript2018. If not, see
% <https://creativecommons.org/licenses/by-nc-sa/4.0/legalcode>.
%
% Copyright TenSing Moers and all whose work and <3 went in this project.

% Creates a list highlightcolors containing all colors for coloring characters.
% The list textcolors contains the corresponding text colors

\newnumberedlist{highlightcolors}
\addtonumberedlist[highlightcolors]{colora}
\addtonumberedlist[highlightcolors]{colorb}
\addtonumberedlist[highlightcolors]{colorc}
\addtonumberedlist[highlightcolors]{colord}
\addtonumberedlist[highlightcolors]{colore}
\addtonumberedlist[highlightcolors]{colorf}
\addtonumberedlist[highlightcolors]{colorg}
\addtonumberedlist[highlightcolors]{colorh}
\addtonumberedlist[highlightcolors]{colori}
\addtonumberedlist[highlightcolors]{colorj}
\addtonumberedlist[highlightcolors]{colork}
\addtonumberedlist[highlightcolors]{colorl}
\addtonumberedlist[highlightcolors]{colorm}
\addtonumberedlist[highlightcolors]{colorn}
\addtonumberedlist[highlightcolors]{coloro}
\addtonumberedlist[highlightcolors]{colorp}
\addtonumberedlist[highlightcolors]{colorq}
\addtonumberedlist[highlightcolors]{colorr}
\addtonumberedlist[highlightcolors]{colors}
\addtonumberedlist[highlightcolors]{colort}
\addtonumberedlist[highlightcolors]{coloru}
\addtonumberedlist[highlightcolors]{colorv}
\addtonumberedlist[highlightcolors]{colorw}
\addtonumberedlist[highlightcolors]{colorx}
\addtonumberedlist[highlightcolors]{colory}
\addtonumberedlist[highlightcolors]{colorz}
\addtonumberedlist[highlightcolors]{colorz}

\newnumberedlist{textcolors}
\addtonumberedlist[textcolors]{black} % Mila Monolog,    Plakat "1984"
\addtonumberedlist[textcolors]{black} % Mila Dialog,     Spielzeug
\addtonumberedlist[textcolors]{black} % Albert Einstein, Knall
\addtonumberedlist[textcolors]{white} % Wilma,           Lichtblitz
\addtonumberedlist[textcolors]{white} % Wilhelmina,      Notizblock für Albert
\addtonumberedlist[textcolors]{white} % Salome,          Verkaufsstand in der Wutwelt
\addtonumberedlist[textcolors]{white} % Cosma,           Müll
\addtonumberedlist[textcolors]{white} % Wilmas Mutter,   Plakat mit Wutpunktestand
\addtonumberedlist[textcolors]{white} % Leon,            Computerstimme aus dem Off
\addtonumberedlist[textcolors]{white} % Lena,            Türklingel
\addtonumberedlist[textcolors]{white} % Charlotte,       Tür?
\addtonumberedlist[textcolors]{white} % Sophia,          Keksdose
\addtonumberedlist[textcolors]{white} % Emma,            Banane
\addtonumberedlist[textcolors]{white} % Aaron,           Glas Orangensaft
\addtonumberedlist[textcolors]{white} % Cleo,            Medizin
\addtonumberedlist[textcolors]{white} % Theo,            Verbandszeug
\addtonumberedlist[textcolors]{white} % Pikachu,         Kleidung
\addtonumberedlist[textcolors]{white} %                  Klopapier
\addtonumberedlist[textcolors]{white} %                  Feuchttücher
\addtonumberedlist[textcolors]{black} %                  Desinfektionsmittel
\addtonumberedlist[textcolors]{black} %                  Strandhandtuch
\addtonumberedlist[textcolors]{white} %                  Meeresrauschen
\addtonumberedlist[textcolors]{white} %                  Einspieler: "Los Pikachu"
\addtonumberedlist[textcolors]{white} %                  Sturm
\addtonumberedlist[textcolors]{black} %                  magischer Kuchen
\addtonumberedlist[textcolors]{black} %                  Klingeln eines Weckers
\addtonumberedlist[textcolors]{black} %                  Bett

% number of the next color to use when none is specified.
\newcounter{colornumber}
\stepcounter{colornumber}\stepcounter{colornumber}

\ExplSyntaxOn
\seq_new:N \g_luke_listofpersons_seq
\seq_new:N \l_luke_listofpersons_seq 


\NewDocumentCommand{\addperson}{m}{%
  \seq_gput_right:Nn \g_luke_listofpersons_seq {#1}
  \seq_gremove_duplicates:N \g_luke_listofpersons_seq
}


\NewDocumentCommand{\addpersonlocal}{m}{%
  \seq_gput_right:Nn \l_luke_listofpersons_seq {#1}
}

\cs_new:Npn \IfPersonCalledAlreadyF #1#2 {%
  \seq_if_in:NnF \l_luke_listofpersons_seq {#1} {#2}
}

\NewDocumentCommand{\DisplayPersons}{}{%
  \seq_clear:N \l_luke_listofpersons_seq
  \group_begin:
  \seq_clear:N \l_tmpa_seq
  \seq_map_inline:Nn \g_luke_listofpersons_seq {%
    \IfRefUndefinedExpandable{##1\thesection}{}{
      \seq_put_right:Nn \l_tmpa_seq {\use:c{##1h}}
     }
   }
   \seq_if_empty:NF \l_tmpa_seq {
     \PrePersonList
     \seq_use:Nn \l_tmpa_seq {,~} % this is the delimiter
     \PostPersonList
   }
   \group_end:

}
\ExplSyntaxOff

\NewDocumentCommand{\PostPersonList}{}{%
  \bigskip%

}

\NewDocumentCommand{\displayindividualperson}{m}{%
  \textbf{#1}%
}

\NewDocumentCommand{\PrePersonList}{}{%
  {\large \bfseries Charaktere:}%
}

% \NewPerson: Creates a new Person with its respective commands
% First Argument:  The Person's full name (as shown in the document)
% Second Argument: The Person's short name (used for the command creation)
% Third Argument:  The Person's highlight color (used for highlighting and 
%                  colored boxes).
% Fourth Argument: The Person's text color (used for displaying the person's
%                  name on the colored background).
\makeatletter
\NewDocumentCommand{\NewPerson}{m+m+m+m}{%
\typeout{Creating person #1...}
  % Add this person to the global list
  \addperson{#2}%
  % Now define the personal \...x command 
  \expandafter\NewDocumentCommand\csname #2\endcsname{}{%
    \textsc{#1} %
  }
  \expandafter\NewDocumentCommand\csname #2x\endcsname{+m}{%
    %Check if the person has been called in the local section already
    \IfPersonCalledAlreadyF{#2}{%
      \addpersonlocal{#2}
      % Add the personal to the local list, i.e. per section
      % Check whether the label has been defined already
        \protected@edef\@currentlabel{\thesection.#2}\label{#2\thesection}
    }%
    \speechbox{#1}{#3}{#4}{##1}%
  } % defines \<charactername>x-command for speech
  \expandafter\NewDocumentCommand\csname #2h\endcsname{}{%
  %Check if the person has been called in the local section already
    \IfPersonCalledAlreadyF{#2}{%
      \addpersonlocal{#2}
      % Add the personal to the local list, i.e. per section
      % Check whether the label has been defined already
        \protected@edef\@currentlabel{\thesection.#2}\label{#2\thesection}
    }%
    \boxcommand{#3}{\strut\textcolor{#4}{\textsc{#1}}}
  } % defines \<charactername>h-command for inline highlighting
}% End of \NewPerson
\makeatother

% \NewPerson: Creates a new Person with its respective commands
% First Argument:  The Person's full name (as shown in the document)
% Second Argument: The Person's short name (used for the command creation)
% Third Argument:  The Person's highlight color (used for highlighting and 
%                  colored boxes).
% Fourth Argument: The Person's text color (used for displaying the person's
%                  name on the colored background).
\makeatletter
\NewDocumentCommand{\NewPersonAutoColor}{m+m}{%
\typeout{Creating person #1...}
  % Add this person to the global list
  \addperson{#2}%
  % Now define the personal \...x command 
  \expandafter\NewDocumentCommand\csname #2\endcsname{}{%
    \textsc{#1} %
  }
  \expandafter\NewDocumentCommand\csname #2x\endcsname{+m}{%
    %Check if the person has been called in the local section already
    \IfPersonCalledAlreadyF{#2}{%
      \addpersonlocal{#2}
      % Add the personal to the local list, i.e. per section
      % Check whether the label has been defined already
        \protected@edef\@currentlabel{\thesection.#2}\label{#2\thesection}
    }%
    \speechbox{#1}{#3}{#4}{##1}%
  } % defines \<charactername>x-command for speech
  \expandafter\NewDocumentCommand\csname #2h\endcsname{}{%
  %Check if the person has been called in the local section already
    \IfPersonCalledAlreadyF{#2}{%
      \addpersonlocal{#2}
      % Add the personal to the local list, i.e. per section
      % Check whether the label has been defined already
        \protected@edef\@currentlabel{\thesection.#2}\label{#2\thesection}
    }%
    \boxcommand{#3}{\strut\textcolor{#4}{\textsc{#1}}}
  } % defines \<charactername>h-command for inline highlighting
}% End of \NewPerson
\makeatother