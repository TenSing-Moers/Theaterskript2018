% This file is part of TenSing-Moers/Theaterskript2018.
%
% TenSing-Moers/Theaterskript2018 is free content: you can redistribute and/or
% modify it under the terms of the cc-by-nc-sa (Creative Commons
% Attribution-NonCommercial-ShareAlike) as released by the
% Creative Commons organisation, version 4.0.
%
% TenSing-Moers/Theaterskript2018 is distributed in the hope that it will be useful,
% but without any warranty.
%
% You should have received a copy of the cc-by-nc-sa-license along
% with this copy of TenSing-Moers/Theaterkskript2018. If not, see
% <https://creativecommons.org/licenses/by-nc-sa/4.0/legalcode>.
%
% Copyright TenSing Moers and all whose work and <3 went in this project.
%Zwei Farben, ein Grauton und Schwarz
\definecolor{grey}{RGB}{100, 100, 100}
\definecolor{black}{RGB}{0, 0, 0}

% Neues Farbschema, basierend auf Regenbogen-Farbschema
\definecolor{colora}{RGB}{252,234,16}
\definecolor{colorb}{RGB}{249,178,51}
\definecolor{colorc}{RGB}{243,146,0}
\definecolor{colord}{RGB}{233,78,27}
\definecolor{colore}{RGB}{227,6,19}
\definecolor{colorf}{RGB}{163,25,91}
\definecolor{colorg}{RGB}{230,0,126}
\colorlet{colorh}{DeepPink1}
\colorlet{colori}{Magenta1}
\colorlet{colorj}{Magenta3}

\definecolor{colork}{RGB}{149,27,129}
\definecolor{colorl}{RGB}{102,36,131}
\definecolor{colorm}{RGB}{41,35,92}
\definecolor{colorn}{RGB}{45,46,131}
\definecolor{coloro}{RGB}{29,113,184}
\definecolor{colorp}{RGB}{54,169,225}
\definecolor{colorq}{RGB}{0,161,154}
\definecolor{colorr}{RGB}{47,172,102}
\definecolor{colors}{RGB}{0,102,51}
\definecolor{colort}{RGB}{58,170,53}
\definecolor{coloru}{RGB}{149,193,31}
\definecolor{colorv}{RGB}{177,127,74}
\definecolor{colorw}{RGB}{125,78,36}
\definecolor{colorx}{RGB}{67,41,24}
\colorlet{colory}{Ivory2}
\colorlet{colorz}{Ivory4}

\colorlet{milamc}{colora}
\colorlet{miladc}{colorb}
\colorlet{albertc}{colorc}
\colorlet{wilmac}{colord}
\colorlet{wilhelminac}{colore}
\colorlet{salomec}{colorf}
\colorlet{cosmac}{colorg}
\colorlet{mutterc}{colorh}
\colorlet{leonc}{colori}
\colorlet{leac}{colorj}
\colorlet{charlottec}{colork}
\colorlet{sophiac}{colorl}
\colorlet{emmac}{colorm}
\colorlet{aaronc}{colorn}
\colorlet{cleoc}{coloro}
\colorlet{theoc}{colorp}
\colorlet{pikachuc}{colorq}